%!TEX root = ./main.tex
%!TEX encoding = UTF-8 Unicode

%----------------------------------------
% Informations
%----------------------------------------

\title{Métaheuristiques Hybrides pour le problème de couverture par ensembles}
\author{Maxime Pinard}
\date{2020}

%----------------------------------------
% Covers configuration
%----------------------------------------

\setutbmfrontillustration{IRIMAS}
\setutbmfrontillustrationwidth{0.9\paperwidth}
\setutbmfrontillustrationyshift{-2cm}

\setutbmtitle{Métaheuristiques Hybrides pour le problème de couverture par ensembles}
\setutbmsubtitle{Rapport de stage ST50 / 8INF859 - P2020}

\setutbmstudent{Maxime Pinard}
\setutbmstudentdepartment{Département Informatique}
\setutbmstudentpathway{Filière Interaction Imagerie Réalité Virtuelle}
\setuqacstudentdepartment{Département d'Informatique et de Mathématique}
\setuqacstudentpathway{Maîtrise en Informatique Profil Professionnel}

\setutbmcompany{Institut de Recherche en Informatique, Mathématiques, Automatique et Signal}
\setutbmcompanyaddress{12 rue des Frères Lumière\\68 093 MULHOUSE Cedex}
\setutbmcompanywebsite{\href{https://www.irimas.uha.fr/}{\color{utbm_cover_main_shadow_text}{https://www.irimas.uha.fr/}}}

\setutbmcompanytutor{Laurent Moalic}
\setutbmschooltutor{Fabrice Lauri}
\setuqacschooltutor{Sara Séguin}

\setutbmkeywords{
	recherche \textendash{}	mathématiques \textendash{} informatique \textendash{} optimisation \textendash{} métaheuristique \textendash{} problème de couverture par ensembles
}
\setutbmabstract{
	Le Unicost Set Cover Problem (USCP), est un problème NP-complet qui fait partie des grands classiques de l'optimisation. Les chercheurs de l'Institut de Recherche en Informatique, Mathématiques, Automatique et Signal (IRIMAS) s'intéressent notamment aux approches heuristiques permettant de résoudre de tels problèmes. Une approche utilisant un algorithme mémétique reposant sur une population réduite à deux individus a récemment obtenu de très bon résultats sur le problème NP-complet bien connu de coloration de graphes et laisse supposer que cette démarche pourrait se révéler particulièrement efficace sur d'autres types de problèmes, tels que le USCP.
	\vspace{10pt}\\
	L'objectif principal du stage était de développer une approche mémétique semblable, reposant sur une population réduite à deux individus, qui soit efficace pour la résolution du USCP. Après une revue de la littérature, l'algorithme Row Weighting Local Search (RWLS) a été sélectionné comme recherche locale de notre hybridation, il a été implémenté, testé et soumis a des benchmarks. Une première version simple de l'algorithme mémétique a été conçue et implémentée, puis une seconde version ajoutant le croisement des poids utilisés en interne par RWLS, et enfin une dernière version nommée Memetic Algorithm for Set Covering (MASC) qui intègre une variation dynamique du nombre d'étapes de RWLS réalisées à chaque génération.
	\vspace{10pt}\\
	Le développement de l'algorithme mémétique a demandé le test de plusieurs idées et stratégies ainsi que l'implémentation de nombreux opérateurs de croisement des solutions et des poids de RWLS. L'efficacité de chaque modification a pu être précisément évaluée par des benchmarks réalisés sur le cluster HPC du méso-centre de l'Université de Strasbourg jusqu'à la version finale de l'algorithme qui a mené à une soumission à la 14\textsuperscript{th} Learning and Intelligent OptimizatioN Conference (LION14) avec l'amélioration de 8 Best Known Solution (BKS) des instances classiques de benchmark.
}

%----------------------------------------
% Configuration
%----------------------------------------

% reduce underscores size
\renewcommand{\_}{\textscale{.5}{\textunderscore}}

% remove chapter from equations numbers
\counterwithout{equation}{chapter}

%----------------------------------------
% Figures
%----------------------------------------

% Common file
%!TEX encoding = UTF-8 Unicode

\usetikzlibrary{shapes}
\usetikzlibrary{arrows.meta}
\usetikzlibrary{calc}

\definecolor{bg_color}{RGB}{250,250,229}

\colorlet{color1}{cyan!50}
\colorlet{color2}{red!30!green!40}
\colorlet{color3}{orange!50}
\colorlet{color4}{violet!60!blue!55}

\definecolor{Cblue}{RGB}{38,75,150}
\definecolor{Cgreen}{RGB}{39,179,118}
\definecolor{Cdarkgreen}{RGB}{0,111,60}
\definecolor{Corange}{RGB}{249,167,62}
\definecolor{Cred}{RGB}{191,33,47}

\newganttlinktype{bartobardown}{
	\ganttsetstartanchor{south east}
	\ganttsetendanchor{north west}
	\draw [/pgfgantt/link] (\xLeft, \yUpper) -- (\xRight, \yLower);
}
\newganttlinktype{bartobarup}{
	\ganttsetstartanchor{north east}
	\ganttsetendanchor{south west}
	\draw [/pgfgantt/link] (\xLeft, \yUpper) -- (\xRight, \yLower);
}
\newganttlinktype{milestonetobardown}{
	\ganttsetstartanchor{south}
	\ganttsetendanchor{north west}
	\draw [/pgfgantt/link] (\xLeft, \yUpper) -- (\xRight, \yLower);
}
\newganttlinktype{bartomilestonedown}{
	\ganttsetstartanchor{south east}
	\ganttsetendanchor{north}
	\draw [/pgfgantt/link] (\xLeft, \yUpper) -- (\xRight, \yLower);
}


% Figures folder
\graphicspath{{../figures/}}

% Figures counting
\counterwithout{figure}{chapter}

%----------------------------------------
% Tables
%----------------------------------------

% Common file
%!TEX encoding = UTF-8 Unicode


% Table counting
\counterwithout{table}{chapter}

%----------------------------------------
% Plots
%----------------------------------------

%\pgfplotsset{
%  table/search path={../plots},
%}

%----------------------------------------
% Boxes
%----------------------------------------

\tcbuselibrary{most}
\newtcolorbox{infobox}[2][]{
	breakable,
	colback=gray!5!,
	colframe=black!75!white,
	arc=0pt,
	title=#2,
	#1
}

%----------------------------------------
% Commands
%----------------------------------------

\newcommand{\solver}{\textit{solver}}
\newcommand{\printer}{\textit{printer}}
\newcommand{\common}{\textit{common}}
