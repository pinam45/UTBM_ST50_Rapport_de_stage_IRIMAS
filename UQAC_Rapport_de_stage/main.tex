%!TEX program = lualatex
%!TEX encoding = UTF-8 Unicode
\PassOptionsToPackage{french}{babel}
\PassOptionsToPackage{french}{translator}
\documentclass[a4paper,11pt,twoside,french,report]{../common/simplem}
%!TEX root = ./main.tex
%!TEX encoding = UTF-8 Unicode

% Pro­gram­ming fa­cil­i­ties
\usepackage{etoolbox}
\usepackage{ifxetex}
\usepackage{ifluatex}

% Encoding
\usepackage[T1]{fontenc}
\ifboolexpr{bool{xetex} or bool{luatex}}{%
	\usepackage{fontspec}
}{%
	\usepackage[utf8]{inputenc}
}

% General purpose
\usepackage{tcolorbox}

% Mathematics
\usepackage{amsmath}
\usepackage{amssymb}
\usepackage{mathrsfs}
\usepackage{amsthm}
\usepackage{dsfont}
\usepackage{braket}
\usepackage{stmaryrd}

% Tables
\usepackage{array}
\usepackage{tabularx}
\usepackage{longtable}
\usepackage{tabu}
\usepackage{booktabs}
\usepackage{multirow}
\usepackage{makecell}
\usepackage{blkarray}

% Figures
\usepackage[mode=tex]{standalone}
\usepackage{import}
\usepackage{float}
\usepackage[justification=centering]{caption}

% PGF-TikZ
\usepackage{pgf}
\usepackage{pgfplots}
\pgfplotsset{compat=1.16}
\usepackage{tikz}
\usepackage{tikzpeople}
\usepackage{pgf-umlsd}
\usepackage{pgfgantt}

%!TEX encoding = UTF-8 Unicode

\makeatletter

%----------------------------------------
% Required packages
%----------------------------------------

\usepackage{fontspec} % For Tahoma font
\usepackage{xcolor}
\usepackage{tikz}

%----------------------------------------
% Colors definitions
%----------------------------------------

\definecolor{utbm_cover_background}{RGB}{255,255,255}

\definecolor{utbm_cover_univname}{RGB}{0,0,0}
\definecolor{utbm_cover_univname_text}{RGB}{255,255,255}

\definecolor{utbm_cover_title}{RGB}{125,92,139}
\definecolor{utbm_cover_title_text}{RGB}{255,255,255}

\definecolor{utbm_cover_subtitle}{RGB}{79,78,85}
\definecolor{utbm_cover_subtitle_text}{RGB}{255,255,255}

\definecolor{utbm_cover_main}{RGB}{217,217,87}
\definecolor{utbm_cover_main_text}{RGB}{0,0,0}
\definecolor{utbm_cover_main_shadow_text}{RGB}{153,153,153}

%----------------------------------------
% Font definition
%----------------------------------------

\newfontfamily\tahomafont{Tahoma}

%----------------------------------------
% Default values
%----------------------------------------

%-----
% Default texts and images not related to the student or the company
\newcommand{\@defaultutbmschoolname}{{\bfseries UNIVERSITÉ DE TECHNOLOGIE} DE BELFORT-MONTBÉLIARD}
\newcommand{\@defaultuqacschoolname}{{\bfseries UNIVERSITÉ DU QUÉBEC} À CHICOUTIMI}
\newcommand{\@defaultutbmcompanytutortext}{\iflanguage{french}{Tuteur en entreprise}{Company tutor}}
\newcommand{\@defaultutbmschooltutortext}{\iflanguage{french}{Suiveur UTBM}{UTBM tutor}}
\newcommand{\@defaultuqacschooltutortext}{\iflanguage{french}{Suiveur UQAC}{UQAC tutor}}
\newcommand{\@defaultutbmkeywordstext}{\iflanguage{french}{Mots clefs}{Keywords}}
\newcommand{\@defaultutbmabstracttext}{\iflanguage{french}{Résumé}{Abstract}}
\newcommand{\@defaultutbmschoollogo}{utbm_logo}
\newcommand{\@defaultuqacschoollogo}{uqac_logo}

%-----
% Default texts and images related to the student or the company
\newcommand{\@defaultutbmfrontillustration}{utbm_default_illustration}
\newcommand{\@defaultutbmfrontillustrationwidth}{width=\paperwidth}
\newcommand{\@defaultutbmfrontillustrationyshift}{0pt}
\newcommand{\@defaultutbmtitle}{\iflanguage{french}{Titre}{Title}}
\newcommand{\@defaultutbmsubtitle}{\iflanguage{french}{Sous-titre}{Subtitle}}
\newcommand{\@defaultutbmstudent}{\iflanguage{french}{Nom Étudiant}{Student name}}
\newcommand{\@defaultutbmstudentdepartment}{\iflanguage{french}{Département UTBM}{UTBM department}}
\newcommand{\@defaultuqacstudentdepartment}{\iflanguage{french}{Département UQAC}{UQAC department}}
\newcommand{\@defaultutbmstudentpathway}{\iflanguage{french}{Filière UTBM}{UTBM pathway}}
\newcommand{\@defaultuqacstudentpathway}{\iflanguage{french}{Filière UQAC}{UQAC pathway}}
\newcommand{\@defaultutbmcompany}{\iflanguage{french}{Entreprise}{Company}}
\newcommand{\@defaultutbmcompanyaddress}{\iflanguage{french}{Adresse}{Address}}
\newcommand{\@defaultutbmcompanywebsite}{\iflanguage{french}{Site web}{Website}}
\newcommand{\@defaultutbmcompanytutor}{\iflanguage{french}{Nom tuteur en entreprise}{Company tutor name}}
\newcommand{\@defaultutbmschooltutor}{\iflanguage{french}{Nom suiveur UTBM}{UTBM tutor name}}
\newcommand{\@defaultuqacschooltutor}{\iflanguage{french}{Nom suiveur UQAC}{UQAC tutor name}}
\newcommand{\@defaultutbmkeywords}{\iflanguage{french}{Liste de mots clefs}{List of keywords}}
\newcommand{\@defaultutbmabstract}{\iflanguage{french}{Texte du résumé}{Abstract text}}

%----------------------------------------
% Configuration commands
%----------------------------------------
% Commands to set the texts and images not related to the student or the company
% Default configuration for English and French languages, use the commands for other languages

%-----
% Set the UTBM school name
% \setutbmschoolname{name}
\newcommand{\setutbmschoolname}[1]{\renewcommand{\@utbmschoolname}{#1}}
\newcommand{\@utbmschoolname}{\@defaultutbmschoolname}

%-----
% Set the UQAC school name
% \setuqacschoolname{name}
\newcommand{\setuqacschoolname}[1]{\renewcommand{\@uqacschoolname}{#1}}
\newcommand{\@uqacschoolname}{\@defaultuqacschoolname}

%-----
% Set the text for the company tutor
% \setutbmcompanytutortext{text}
\newcommand{\setutbmcompanytutortext}[1]{\renewcommand{\@utbmcompanytutortext}{#1}}
\newcommand{\@utbmcompanytutortext}{\@defaultutbmcompanytutortext}

%-----
% Set the text for the UTBM school tutor
% \setutbmschooltutortext{text}
\newcommand{\setutbmschooltutortext}[1]{\renewcommand{\@utbmschooltutortext}{#1}}
\newcommand{\@utbmschooltutortext}{\@defaultutbmschooltutortext}

%-----
% Set the text for the UQAC school tutor
% \setuqacschooltutortext{text}
\newcommand{\setuqacschooltutortext}[1]{\renewcommand{\@uqacschooltutortext}{#1}}
\newcommand{\@uqacschooltutortext}{\@defaultuqacschooltutortext}

%-----
% Set the text for the keywords
% \setutbmkeywordstext{text}
\newcommand{\setutbmkeywordstext}[1]{\renewcommand{\@utbmkeywordstext}{#1}}
\newcommand{\@utbmkeywordstext}{\@defaultutbmkeywordstext}

%-----
% Set the text for the abstract
% \setutbmabstracttext{text}
\newcommand{\setutbmabstracttext}[1]{\renewcommand{\@utbmabstracttext}{#1}}
\newcommand{\@utbmabstracttext}{\@defaultutbmabstracttext}

%-----
% Set the UTBM school logo (the file name can be as in the \includegraphics command)
% \setutbmschoollogo{filename}
\newcommand{\setutbmschoollogo}[1]{\renewcommand{\@utbmschoollogo}{#1}}
\newcommand{\@utbmschoollogo}{\@defaultutbmschoollogo}

%-----
% Set the UQAC school logo (the file name can be as in the \includegraphics command)
% \setuqacschoollogo{filename}
\newcommand{\setuqacschoollogo}[1]{\renewcommand{\@uqacschoollogo}{#1}}
\newcommand{\@uqacschoollogo}{\@defaultuqacschoollogo}

%----------------------------------------
% Informations setting commands
%----------------------------------------
% Commands to set the texts and images related to the student or the company
% Default configuration can be used to see where will be displayed each information
% but the commands should be used to set the cover informations

%-----
% Set the illustration figure on the front page (the file name can be as in the \includegraphics command)
% \setutbmfrontillustration{filename}
\newcommand{\setutbmfrontillustration}[1]{\renewcommand{\@utbmfrontillustration}{#1}}
\newcommand{\@utbmfrontillustration}{\@defaultutbmfrontillustration}

%-----
% Set the illustration figure options (options for the \includegraphics command)
% \setutbmfrontillustrationwidth{options}
\newcommand{\setutbmfrontillustrationwidth}[1]{\renewcommand{\@utbmfrontillustrationwidth}{#1}}
\newcommand{\@utbmfrontillustrationwidth}{\@defaultutbmfrontillustrationwidth}

%-----
% Set the illustration figure y shift from the top of the page
% \setutbmfrontillustrationyshift{yshift}
\newcommand{\setutbmfrontillustrationyshift}[1]{\renewcommand{\@utbmfrontillustrationyshift}{#1}}
\newcommand{\@utbmfrontillustrationyshift}{\@defaultutbmfrontillustrationyshift}

%-----
% Set the title
% \setutbmtitle{title}
\newcommand{\setutbmtitle}[1]{\renewcommand{\@utbmtitle}{#1}}
\newcommand{\@utbmtitle}{\@defaultutbmtitle}

%-----
% Set the subtitle
% \setutbmsubtitle{subtitle}
\newcommand{\setutbmsubtitle}[1]{\renewcommand{\@utbmsubtitle}{#1}}
\newcommand{\@utbmsubtitle}{\@defaultutbmsubtitle}

%-----
% Set the student
% \setutbmstudent{name}
\newcommand{\setutbmstudent}[1]{\renewcommand{\@utbmstudent}{#1}}
\newcommand{\@utbmstudent}{\@defaultutbmstudent}

%-----
% Set the student UTBM department
% \setutbmstudentdepartment{department}
\newcommand{\setutbmstudentdepartment}[1]{\renewcommand{\@utbmstudentdepartment}{#1}}
\newcommand{\@utbmstudentdepartment}{\@defaultutbmstudentdepartment}

%-----
% Set the student UQAC department
% \setuqacstudentdepartment{department}
\newcommand{\setuqacstudentdepartment}[1]{\renewcommand{\@uqacstudentdepartment}{#1}}
\newcommand{\@uqacstudentdepartment}{\@defaultuqacstudentdepartment}

%-----
% Set the student UTBM pathway
% \setutbmstudentpathway{pathway}
\newcommand{\setutbmstudentpathway}[1]{\renewcommand{\@utbmstudentpathway}{#1}}
\newcommand{\@utbmstudentpathway}{\@defaultutbmstudentpathway}

%-----
% Set the student UQAC pathway
% \setuqacstudentpathway{pathway}
\newcommand{\setuqacstudentpathway}[1]{\renewcommand{\@uqacstudentpathway}{#1}}
\newcommand{\@uqacstudentpathway}{\@defaultuqacstudentpathway}

%-----
% Set the company
% \setutbmcompany{name}
\newcommand{\setutbmcompany}[1]{\renewcommand{\@utbmcompany}{#1}}
\newcommand{\@utbmcompany}{\@defaultutbmcompany}

%-----
% Set the company address
% \setutbmcompanyaddress{address}
\newcommand{\setutbmcompanyaddress}[1]{\renewcommand{\@utbmcompanyaddress}{#1}}
\newcommand{\@utbmcompanyaddress}{\@defaultutbmcompanyaddress}

%-----
% Set the company website
% \setutbmcompanywebsite{website}
\newcommand{\setutbmcompanywebsite}[1]{\renewcommand{\@utbmcompanywebsite}{#1}}
\newcommand{\@utbmcompanywebsite}{\@defaultutbmcompanywebsite}

%-----
% Set the company tutor
% \setutbmcompanytutor{tutor}
\newcommand{\setutbmcompanytutor}[1]{\renewcommand{\@utbmcompanytutor}{#1}}
\newcommand{\@utbmcompanytutor}{\@defaultutbmcompanytutor}

%-----
% Set the UTBM school tutor
% \setutbmschooltutor{tutor}
\newcommand{\setutbmschooltutor}[1]{\renewcommand{\@utbmschooltutor}{#1}}
\newcommand{\@utbmschooltutor}{\@defaultutbmschooltutor}

%-----
% Set the UQAC school tutor
% \setuqacschooltutor{tutor}
\newcommand{\setuqacschooltutor}[1]{\renewcommand{\@uqacschooltutor}{#1}}
\newcommand{\@uqacschooltutor}{\@defaultuqacschooltutor}

%-----
% Set the keywords
% \setutbmkeywords{keywords}
\newcommand{\setutbmkeywords}[1]{\renewcommand{\@utbmkeywords}{#1}}
\newcommand{\@utbmkeywords}{\@defaultutbmkeywords}

%-----
% Set the abstract
% \setutbmabstract{abstract}
\newcommand{\setutbmabstract}[1]{\renewcommand{\@utbmabstract}{#1}}
\newcommand{\@utbmabstract}{\@defaultutbmabstract}

%----------------------------------------
% Cover generation commands
%----------------------------------------

%-----
% Make the front cover
% \makeutbmfrontcover[shift]
\newcommand{\makeutbmfrontcover}[1][(0,0)]{
	\clearpage
	\thispagestyle{empty}
	\parindent0pt
	\begin{tikzpicture}[remember picture,overlay]
		\tikzset{
			utbm text/.style={
				text badly ragged,
				inner xsep=1cm,
				outer sep=0pt,
				text width=\paperwidth-2cm,
				anchor=north,
			},
			utbm rectangle/.style={
				rectangle,
				inner sep=0pt,
				outer sep=0pt,
				anchor=north,
				minimum width=\paperwidth,
			}
		}

		\node[
			shift={#1},
			inner sep=0pt,
			anchor=north,
			minimum width=\paperwidth,
			minimum height=\paperheight,
		]
		(anchor) at (current page.north)
		{};

		\node[
			utbm rectangle,
			fill=utbm_cover_background,
			minimum height=\paperheight,
		]
		(background) at (anchor.north)
		{};

		% Hide illustration if too hight
		\node[
			yshift=-8.6cm,
			utbm rectangle,
			fill=utbm_cover_background,
			minimum height=\paperheight-8.6cm,
		]
		(background2) at (anchor.north)
		{};

		\node[
			yshift=\@utbmfrontillustrationyshift,
			inner sep=0pt,
			anchor=north,
		]
		(illustration) at (anchor.north)
		{\includegraphics[width=\@utbmfrontillustrationwidth]{\@utbmfrontillustration}};

		\node[
			yshift=-7.6cm,
			utbm text,
			fill=utbm_cover_univname,
			text=utbm_cover_univname_text,
			minimum height=0.8cm,
		]
		(schoolname) at (anchor.north)
		{\tahomafont\fontsize{14.4}{17.2}\selectfont {\@utbmschoolname}\\{\@uqacschoolname}};

		\node[utbm text,
			fill=utbm_cover_title,
			text=utbm_cover_title_text,
			minimum height=2.8cm,
		]
		(title) at (schoolname.south)
		{\tahomafont\fontsize{24}{29}\selectfont \bfseries \@utbmtitle};

		\node[utbm text,
			fill=utbm_cover_subtitle,
			text=utbm_cover_subtitle_text,
			minimum height=0.8cm,
		]
		(subtitle) at (title.south)
		{\tahomafont\fontsize{12}{14}\selectfont \bfseries \@utbmsubtitle};

		\node[
			utbm rectangle,
			fill=utbm_cover_main,
			minimum height=13.65cm,
		]
		(information) at (subtitle.south)
		{};

		\node[
			yshift=-1.3cm,
			utbm text,
			text=black,
			minimum height=0.8cm,
		]
		(student) at (subtitle.south)
		{\tahomafont\bfseries {\fontsize{18}{22}\selectfont \@utbmstudent}};

		\node[
			yshift=-2.4cm,
			utbm text,
			text=black,
			minimum height=0.8cm,
		]
		(studentutbminfo) at (subtitle.south)
		{\tahomafont\bfseries\fontsize{10}{12}\selectfont{\fontsize{12}{14}\selectfont UTBM}\\
		\@utbmstudentdepartment\\
		{\color{utbm_cover_main_shadow_text}\@utbmstudentpathway}};

		\node[
			yshift=-2.4cm,
			xshift=10cm,
			utbm text,
			text=black,
			minimum height=0.8cm,
		]
		(studentuqacinfo) at (subtitle.south)
		{\tahomafont\bfseries\fontsize{10}{12}\selectfont{\fontsize{12}{14}\selectfont UQAC}\\
		\@uqacstudentdepartment\\
		{\color{utbm_cover_main_shadow_text}\@uqacstudentpathway}};

		\node[
			yshift=-6cm,
			utbm text,
			text=utbm_cover_main_text,
			minimum height=0.8cm,
		]
		(company) at (subtitle.south)
		{\tahomafont\bfseries {\fontsize{20}{24}\selectfont \@utbmcompany}\\\vspace{0.3cm}
		{\fontsize{12}{14}\selectfont\color{utbm_cover_main_shadow_text} \@utbmcompanyaddress\\\vspace{0.1cm}
		\@utbmcompanywebsite}};

		\node[
			yshift=-10.8cm,
			utbm text,
			text=utbm_cover_main_text,
			minimum height=0.8cm,
		]
		(company_tutor) at (subtitle.south)
		{\tahomafont {\fontsize{12}{14}\selectfont\color{utbm_cover_main_shadow_text} \@utbmcompanytutortext}\\
		\vspace*{0.7mm}{\fontsize{14}{17}\selectfont\bfseries \@utbmcompanytutor}};

		\node[
			yshift=-10.8cm,
			utbm text,
			align=right,
			text=utbm_cover_main_text,
			minimum height=0.8cm,
		]
		(school_tutor) at (subtitle.south)
		{\tahomafont {\fontsize{12}{14}\selectfont\color{utbm_cover_main_shadow_text} \@uqacschooltutortext}\\
		\vspace*{0.7mm}{\fontsize{14}{17}\selectfont\bfseries \@uqacschooltutor}};

		\node[
			yshift=-8.8cm,
			utbm text,
			align=right,
			text=utbm_cover_main_text,
			minimum height=0.8cm,
		]
		(school_tutor) at (subtitle.south)
		{\tahomafont {\fontsize{12}{14}\selectfont\color{utbm_cover_main_shadow_text} \@utbmschooltutortext}\\
		\vspace*{0.7mm}{\fontsize{14}{17}\selectfont\bfseries \@utbmschooltutor}};

		\node[
			yshift=0.7cm,
			xshift=-1cm,
			inner sep=0pt,
			anchor=south east,
		]
		(schoollogo) at (anchor.south east)
		{\includegraphics[width=5.18cm]{\@utbmschoollogo}};

		\node[
			yshift=0.4cm,
			xshift=1cm,
			inner sep=0pt,
			anchor=south west,
		]
		(schoollogo) at (anchor.south west)
		{\includegraphics[width=4.5cm]{\@uqacschoollogo}};


		% \node[
		% 	rectangle,
		% 	red,
		% 	fill=red,
		% 	inner sep=0pt,
		% 	anchor=north,
		% 	minimum width=\paperwidth,
		% 	minimum height=\paperheight,
		% ]
		% (anchor) at (current page.north)
		% {};
	\end{tikzpicture}
	\newpage{}
}

%-----
% Make the back cover
% \makeutbmbackcover[shift]
\newcommand{\makeutbmbackcover}[1][(0,0)]{
	\clearpage \ifodd\value{page}\hbox{}\newpage\fi
	\thispagestyle{empty}
	\parindent0pt
	\begin{tikzpicture}[remember picture,overlay]
		\tikzset{
			utbm text/.style={
				text badly ragged,
				inner xsep=1cm,
				outer sep=0pt,
				text width=\paperwidth-2cm,
			},
			utbm rectangle/.style={
				rectangle,
				inner sep=0pt,
				outer sep=0pt,
				anchor=north,
				minimum width=\paperwidth,
			}
		}

		\node[
			shift={#1},
			inner sep=0pt,
			anchor=north,
			minimum width=\paperwidth,
			minimum height=\paperheight,
		]
		(anchor) at (current page.north)
		{};

		\node[
			utbm rectangle,
			fill=utbm_cover_background,
			minimum height=\paperheight,
		]
		(background) at (anchor.north)
		{};

		\node[
			yshift=-8.4cm,
			utbm rectangle,
			fill=utbm_cover_subtitle,
			minimum height=1cm,
		]
		(line) at (anchor.north)
		{};

		\node[
			utbm rectangle,
			fill=utbm_cover_main,
			minimum height=17.25cm,
		]
		(information) at (line.south)
		{};

		\node[
			yshift=-1.5cm,
			utbm text,
			text=utbm_cover_main_text,
			anchor=north,
			minimum height=0.8cm,
		]
		(keywords) at (anchor.north)
		{\tahomafont\fontsize{12}{14}\selectfont {\bfseries \@utbmkeywordstext}\\
		\hfil\\
		\@utbmkeywords};

		\node[
			yshift=0.25cm,
			utbm text,
			text=utbm_cover_main_text,
			anchor=south,
			minimum height=0.8cm,
		]
		(student) at (line.north)
		{\tahomafont\fontsize{14}{17}\selectfont\bfseries \@utbmstudent};

		\node[
			yshift=0.25cm,
			utbm text,
			align=right,
			text=utbm_cover_main_text,
			anchor=south,
			minimum height=0.8cm,
		]
		(subtitle) at (line.north)
		{\tahomafont\fontsize{14}{17}\selectfont\bfseries \@utbmsubtitle};

		\node[
			yshift=-1.2cm,
			utbm text,
			text justified,
			text=utbm_cover_main_text,
			anchor=north,
			minimum height=0.8cm,
		]
		(abstract) at (information.north)
		{\tahomafont\fontsize{12}{14}\selectfont {\bfseries \@utbmabstracttext}\\
		\hfil\\
		\@utbmabstract};

		\node[
			yshift=0.5cm,
			utbm text,
			text=utbm_cover_main_text,
			anchor=south,
			minimum height=0.8cm,
		]
		(company) at (information.south)
		{\tahomafont\bfseries {\fontsize{20}{24}\selectfont \@utbmcompany}\\\vspace{0.3cm}
		{\fontsize{12}{14}\selectfont\color{utbm_cover_main_shadow_text} \@utbmcompanyaddress\\\vspace{0.1cm}
		\@utbmcompanywebsite}};

		\node[
			yshift=0.5cm,
			xshift=-1cm,
			inner sep=0pt,
			anchor=south east,
		]
		(schoollogo) at (anchor.south east)
		{\includegraphics[width=5.18cm]{\@utbmschoollogo}};

		\node[
			yshift=0.2cm,
			xshift=1cm,
			inner sep=0pt,
			anchor=south west,
		]
		(schoollogo) at (anchor.south west)
		{\includegraphics[width=4.5cm]{\@uqacschoollogo}};
	\end{tikzpicture}
}

\makeatother

%!TEX encoding = UTF-8 Unicode

% Print number with at least 2 digits
\newcommand{\twodigits}[1]{\ifnum#1<10 0\fi\the#1}

% Print email link
\newcommand{\email}[1]{\href{mailto:#1}{#1}}

% instance
\newcommand{\instance}[1]{\texttt{#1}}

% C++
\newcommand{\Cpp}{\texorpdfstring{{C\nolinebreak[4]\hspace{-.05em}\raisebox{.4ex}{\tiny\bf ++}}}{C++}}

% Clear to the next left page
\newcommand*{\cleartoleftpage}{\clearpage\ifodd\value{page}\hbox{}\newpage\fi}

% Create 0-width tikz node inline
\newcommand{\tikzremember}[1]{\makebox[0pt][c]{\tikz[remember picture]\node(#1){};}}

% Custom label
\makeatletter
\newcommand{\customlabel}[2]{\def\@currentlabel{#2}\label{#1}}
\makeatother

% Table plot
\newcommand{\tableplot}[1]{%
	\raisebox{-2pt}{%
		\resizebox{25pt}{9pt}{%
			\begin{tikzpicture}%
				\begin{axis}[%
					ybar,%
					width=1000pt,%
					%height=55pt,%
			  		bar width=0.75,%
					xmin=0,%
					ymin=0,%
					enlargelimits=0,%
			  		axis lines=none,%
			  		xtick=\empty,%
			  		ytick=\empty,%
				]%
				\foreach \x [count=\xi] in {#1}
				{
					\ifthenelse{\isodd\xi}{
						\addplot+[
							black,
							fill=black,
							bar shift=-0.5
						] coordinates{(\xi,\x)};
					}{
						\addplot+[
							gray,
							fill=gray,
							bar shift=-0.5
						] coordinates{(\xi,\x)};
					}
				}
				\end{axis}
			\end{tikzpicture}%
		}%
	}%
}

%!TEX root = ./main.tex
%!TEX encoding = UTF-8 Unicode

%----------------------------------------
% Informations
%----------------------------------------

\title{Métaheuristiques Hybrides pour le problème de couverture par ensembles}
\author{Maxime Pinard}
\date{2020}

%----------------------------------------
% Covers configuration
%----------------------------------------

\setutbmfrontillustration{IRIMAS}
\setutbmfrontillustrationwidth{0.9\paperwidth}
\setutbmfrontillustrationyshift{-2cm}

\setutbmtitle{Métaheuristiques Hybrides pour le problème de couverture par ensembles}
\setutbmsubtitle{Rapport de stage ST50 / 8INF859 - P2020}

\setutbmstudent{Maxime Pinard}
\setutbmstudentdepartment{Département Informatique}
\setutbmstudentpathway{Filière Interaction Imagerie Réalité Virtuelle}
\setuqacstudentdepartment{Département d'Informatique et de Mathématique}
\setuqacstudentpathway{Maîtrise en Informatique Profil Professionnel}

\setutbmcompany{Institut de Recherche en Informatique, Mathématiques, Automatique et Signal}
\setutbmcompanyaddress{12 rue des Frères Lumière\\68 093 MULHOUSE Cedex}
\setutbmcompanywebsite{\href{https://www.irimas.uha.fr/}{\color{utbm_cover_main_shadow_text}{https://www.irimas.uha.fr/}}}

\setutbmcompanytutor{Laurent Moalic}
\setutbmschooltutor{Fabrice Lauri}
\setuqacschooltutor{Sara Séguin}

\setutbmkeywords{
	recherche \textendash{}	mathématiques \textendash{} informatique \textendash{} optimisation \textendash{} métaheuristique \textendash{} problème de couverture par ensembles
}
\setutbmabstract{
	Le problème de « couverture par ensembles », également connu sous le nom de « set covering problem », fait partie des grands classiques de l’optimisation. Au sein de l’institut de recherche IRIMAS, nous nous intéressons notamment aux approches heuristiques permettant de résoudre de tels
	problèmes.
	\vspace{12pt}\\
	Dans le cadre de ce stage, nous nous intéressons au développement de méthodes hybrides de type mémétique, mêlant recherche locale et algorithmes génétiques. Ce type d’hybridation s’est révélé particulièrement efficace pour résoudre des problèmes connus pour être très difficiles, appartenant à la famille des problèmes NP-complets. Dans le cadre de travaux menés conjointement avec l’ENAC de Toulouse, nous avons développé des algorithmes reposant sur des populations réduites à deux individus. Les résultats obtenus sur le problème bien connu de coloration de graphes nous laisse supposer que cette démarche pourrait se révéler particulièrement efficace sur d’autres types de problèmes, tels que le « set cover ».
}

%----------------------------------------
% Configuration
%----------------------------------------

% reduce underscores size
\renewcommand{\_}{\textscale{.5}{\textunderscore}}

%----------------------------------------
% Figures
%----------------------------------------

% Common file
%!TEX encoding = UTF-8 Unicode

\usetikzlibrary{shapes}
\usetikzlibrary{arrows.meta}
\usetikzlibrary{calc}

\definecolor{bg_color}{RGB}{250,250,229}

\colorlet{color1}{cyan!50}
\colorlet{color2}{red!30!green!40}
\colorlet{color3}{orange!50}
\colorlet{color4}{violet!60!blue!55}

\definecolor{Cblue}{RGB}{38,75,150}
\definecolor{Cgreen}{RGB}{39,179,118}
\definecolor{Cdarkgreen}{RGB}{0,111,60}
\definecolor{Corange}{RGB}{249,167,62}
\definecolor{Cred}{RGB}{191,33,47}

\newganttlinktype{bartobardown}{
	\ganttsetstartanchor{south east}
	\ganttsetendanchor{north west}
	\draw [/pgfgantt/link] (\xLeft, \yUpper) -- (\xRight, \yLower);
}
\newganttlinktype{bartobarup}{
	\ganttsetstartanchor{north east}
	\ganttsetendanchor{south west}
	\draw [/pgfgantt/link] (\xLeft, \yUpper) -- (\xRight, \yLower);
}
\newganttlinktype{milestonetobardown}{
	\ganttsetstartanchor{south}
	\ganttsetendanchor{north west}
	\draw [/pgfgantt/link] (\xLeft, \yUpper) -- (\xRight, \yLower);
}
\newganttlinktype{bartomilestonedown}{
	\ganttsetstartanchor{south east}
	\ganttsetendanchor{north}
	\draw [/pgfgantt/link] (\xLeft, \yUpper) -- (\xRight, \yLower);
}


% Figures folder
\graphicspath{{../figures/}}

% Figures counting
\counterwithout{figure}{chapter}

%----------------------------------------
% Tables
%----------------------------------------

% Common file
%!TEX encoding = UTF-8 Unicode

\usetikzlibrary{shapes}
\usetikzlibrary{arrows.meta}
\usetikzlibrary{calc}

\definecolor{bg_color}{RGB}{250,250,229}

\colorlet{color1}{cyan!50}
\colorlet{color2}{red!30!green!40}
\colorlet{color3}{orange!50}
\colorlet{color4}{violet!60!blue!55}

\definecolor{Cblue}{RGB}{38,75,150}
\definecolor{Cgreen}{RGB}{39,179,118}
\definecolor{Cdarkgreen}{RGB}{0,111,60}
\definecolor{Corange}{RGB}{249,167,62}
\definecolor{Cred}{RGB}{191,33,47}

\newganttlinktype{bartobardown}{
	\ganttsetstartanchor{south east}
	\ganttsetendanchor{north west}
	\draw [/pgfgantt/link] (\xLeft, \yUpper) -- (\xRight, \yLower);
}
\newganttlinktype{bartobarup}{
	\ganttsetstartanchor{north east}
	\ganttsetendanchor{south west}
	\draw [/pgfgantt/link] (\xLeft, \yUpper) -- (\xRight, \yLower);
}
\newganttlinktype{milestonetobardown}{
	\ganttsetstartanchor{south}
	\ganttsetendanchor{north west}
	\draw [/pgfgantt/link] (\xLeft, \yUpper) -- (\xRight, \yLower);
}
\newganttlinktype{bartomilestonedown}{
	\ganttsetstartanchor{south east}
	\ganttsetendanchor{north}
	\draw [/pgfgantt/link] (\xLeft, \yUpper) -- (\xRight, \yLower);
}


% Table counting
\counterwithout{table}{chapter}

%----------------------------------------
% Plots
%----------------------------------------

%\pgfplotsset{
%  table/search path={../plots},
%}

%----------------------------------------
% Boxes
%----------------------------------------

\tcbuselibrary{most}
\newtcolorbox{infobox}[2][]{
	breakable,
	colback=gray!5!,
	colframe=black!75!white,
	arc=0pt,
	title=#2,
	#1
}

%----------------------------------------
% Commands
%----------------------------------------

\newcommand{\solver}{\textit{solver}}
\newcommand{\printer}{\textit{printer}}
\newcommand{\common}{\textit{common}}


%----------------------------------------
% Bibliography
%----------------------------------------
\addbibresource{../references/references.bib}
%\nocite{*}

%----------------------------------------
% Glossary
%----------------------------------------
\makeglossaries
%!TEX encoding = UTF-8 Unicode

\newacronym{8INF808}{8INF808}{Informatique appliquée et optimisation}
\newacronym{8INF852}{8INF852}{Métaheuristiques en optimisation}
\newacronym{8INF859}{8INF859}{Stage}
\newacronym{8INF870}{8INF870}{Algorithmique}
\newacronym{AG41}{AG41}{Optimisation et recherche opérationnelle}
\newacronym{API}{API}{Application Programming Interface}
\newacronym{ARCHIMEDE}{ARCHIMEDE}{Archéologie et Histoire de la Méditerranée et de l'Europe}
\newacronym{BETA}{BETA}{Bureau d'Économie Théorique Appliquée}
\newacronym{BKS}{BKS}{Best Known Solution}
\newacronym{CERDACC}{CERDACC}{Centre Européen de Recherche sur les Droits des Accidents Collectifs et Catastrophes}
\newacronym{CFAU}{CFAU}{Centre de Formation d'Apprentis Universitaire}
\newacronym{CLAM}{CLAM}{service de Certifications et Langues par Apprentissage Multimédia}
\newacronym{CREGO}{CREGO}{Centre de Recherche en Gestion des Organisations}
\newacronym{CRESAT}{CRESAT}{Centre de Recherche sur les Économies, les Sociétés, les Arts et les Technologies}
\newacronym{CUFEF}{CUFEF}{Centre Universitaire de Formation des Enseignants et des Formateurs}
\newacronym{DIMACS}{DIMACS}{Center for Discrete Mathematics and Theoretical Computer Science}
\newacronym{DIM}{DIM}{Département d'Informatique et de Mathématique}
\newacronym{DUT}{DUT}{Diplômes Universitaires de Technologie}
\newacronym{EA2019}{EA2019}{14\textsuperscript{th} Biennial International Conference on Artificial Evolution}
\newacronym{ENSCMu}{ENSCMu}{École Nationale Supérieure de Chimie de Mulhouse}
\newacronym{ENSISA}{ENSISA}{École Nationale Supérieure d'Ingénieurs Sud Alsace}
\newacronym{Equip@meso}{Equip@meso}{Equipement d'excellence de calcul intensif de Mésocentres coordonnés}
\newacronym{FIFO}{FIFO}{First In First Out}
\newacronym{FLSH}{FLSH}{Faculté des Lettres, Langues et Sciences Humaines}
\newacronym{FMA}{FMA}{Faculté de Marketing et d'Agrosciences}
\newacronym{FOTI}{FOTI}{Fonctions Optiques et Traitement des Images}
\newacronym{FSESJ}{FSESJ}{Faculté des Sciences Économiques Sociales et Juridiques}
\newacronym{FST}{FST}{Faculté des Sciences et Techniques}
\newacronym{GRE}{GRE}{Laboratoire Gestion des Risques et Environnement}
\newacronym{GVCP}{GVCP}{Graph Vertex Coloring Problem}
\newacronym{HEAD}{HEAD}{Hybrid Evolutionary Algorithm in Duet}
\newacronym{HEA}{HEA}{Hybrid Evolutionary Algorithm}
\newacronym{HPC}{HPC}{High-Performance Computing}
\newacronym{IDE}{IDE}{Environnement de Développement Intégré}
\newacronym{ILLE}{ILLE}{Institut de Recherche en Langues et Littératures Européennes}
\newacronym{IMTI}{IMTI}{Imagerie Microscopique et Traitement d'Images}
\newacronym{IPHC}{IPHC}{Institut Pluridisciplinaire Hubert Curien}
\newacronym{IRIMAS}{IRIMAS}{Institut de Recherche en Informatique, Mathématiques, Automatique et Signal}
\newacronym{IUT}{IUT}{Institut Universitaire de Technologie}
\newacronym{LIMA}{LIMA}{Laboratoire d'Innovation Moléculaire et Applications}
\newacronym{LION14}{LION14}{14\textsuperscript{th} Learning and Intelligent OptimizatioN Conference}
\newacronym{LISEC}{LISEC}{Laboratoire Interuniversitaire des Sciences de l'Éducation et de la Communication}
\newacronym{LMIA}{LMIA}{Laboratoire de Mathématiques, Informatique et Applications}
\newacronym{LPIM}{LPIM}{Laboratoire de Photochimie et d'Ingénierie Macromoléculaire}
\newacronym{LPMT}{LPMT}{Laboratoire de Physique et Mécanique Textiles}
\newacronym{LTO}{LTO}{Link Time Optimization}
\newacronym{LVBE}{LVBE}{Laboratoire Vigne, Biotechnologies et Environnement}
\newacronym{MASC}{MASC}{Memetic Algorithm for Set Covering}
\newacronym{MIAM}{MIAM}{Modélisation et Identification en Automatique et Mécanique}
\newacronym{MIPS}{MIPS}{Modélisation, Intelligence, Processus et Systèmes}
\newacronym{MSD}{MSD}{Modélisation et Science de Données}
\newacronym{NP}{NP}{Nondeterministic Polynomial time}
\newacronym{OCP}{OCP}{Optimal Camera Placement Problem}
\newacronym{OMeGA}{OMeGA}{Optimisation par Métaheuristiques et alGorithmique et modélisAtion}
\newacronym{OpenMP}{OpenMP}{Open Multi-Processing}
\newacronym{PGO}{PGO}{Profile Guided Optimization}
\newacronym{PUX}{PUX}{permutation uniform-like crossover}
\newacronym{ROADEF2020}{ROADEF2020}{21\textsuperscript{ème} congrès de la Société française de recherche opérationnelle et d'aide à la décision}
\newacronym{ROADEF}{ROADEF}{Société française de recherche opérationnelle et d'aide à la décision}
\newacronym{RTe}{RT}{Réseaux et Télécommunications}
\newacronym{RT}{RT}{Recherche Tabou}
\newacronym{RWLS}{RWLS}{Row Weighting Local Search}
\newacronym{S2M}{S2M}{Institut de Science des Matériaux de Mulhouse}
\newacronym{SAGE}{SAGE}{Société, Acteurs, Gouvernement en Europe}
\newacronym{SAT}{SAT}{Boolean satisfiability problem}
\newacronym{SCP}{SCP}{Set Covering Problem}
\newacronym{SERFA}{SERFA}{Service d'Enseignement et de Recherche en Formation pour Adultes}
\newacronym{SIMD}{SIMD}{Single Instruction Multiple Data}
\newacronym{ST50}{ST50}{Projet de fin d'études}
\newacronym{STS}{STS}{Steiner Triple Systems}
\newacronym{TSP}{TSP}{Travelling Salesman Problem}
\newacronym{UHA}{UHA}{Université de Haute Alsace}
\newacronym{UQAC}{UQAC}{Université du Québec à Chicoutimi}
\newacronym{USCP}{USCP}{Unicost Set Covering Problem}
\newacronym{UTBM}{UTBM}{Université de technologie de Belfort Montbéliard}
\newacronym{UwSCP}{USCP}{Unweighted Set Covering Problem}
\newacronym{UX}{UX}{uniform crossover}


%----------------------------------------
% document
%----------------------------------------
\begin{document}
	\makeutbmfrontcover{}
	\makecopirightpage{
		\begin{center}
			\def\arraystretch{1.1}
			\input{../tables/short/report_synoptic}
			\\\hfill\\\hfill\\
			%!TEX encoding = UTF-8 Unicode

\begin{tabularx}{0.9\textwidth}{|X|l|c|}
	\hline
	\multicolumn{3}{|c|}{\cellcolor{gray!30}Autheurs}\\
	\hline
	\multicolumn{1}{|c|}{\cellcolor{gray!30}\textit{Noms}} & \cellcolor{gray!30}\textit{Commentaires} & \cellcolor{gray!30}\textit{Email}\\
	\hline
	Maxime Pinard & Étudiant \acrshort{UTBM} / \acrshort{UQAC} & \email{maxime.pin@live.fr}\\
	\hline
\end{tabularx}

			\\\hfill\\\hfill\\
			%!TEX encoding = UTF-8 Unicode

\begin{tabularx}{0.9\textwidth}{|X|l|c|}
	\hline
	\multicolumn{3}{|c|}{\cellcolor{gray!30}Validateurs}\\
	\hline
	\multicolumn{1}{|c|}{\cellcolor{gray!30}\textit{Noms}} & \cellcolor{gray!30}\textit{Commentaires} & \cellcolor{gray!30}\textit{Email}\\
	\hline
	Fabrice Lauri & Suiveur \acrshort{UTBM} & \email{fabrice.lauri@utbm.fr}\\
	Sara Séguin & Suiveuse \acrshort{UQAC} & \email{sara\_seguin@uqac.ca}\\
	Laurent Moalic & Tuteur en entreprise & \email{laurent.moalic@uha.fr}\\
	\hline
\end{tabularx}

		\end{center}
	}
	\chapter*{Remerciements}\addcontentsline{toc}{chapter}{Remerciements}
		\paragraph*{}
			Je tiens tout particulièrement à remercier mon maître de stage \textbf{Laurent Moalic} pour m'avoir permis de travailler sur un sujet intéressant et formateur. Je le remercie aussi, ainsi que \textbf{Mathieu Brévilliers} et \textbf{Julien Lepagnot} pour leur collaboration et l'aide qu'ils m'ont apporté sur le projet de recherche réalisé durant ce stage.
		\paragraph*{}
			Je tiens aussi à remercier \textbf{Dominique Schmitt} qui a partagé son bureau avec moi, ainsi que \textbf{Julien Kritter}, \textbf{Hojjat Rakhshani}, \textbf{Soheila Ghambari}, \textbf{Mokhtar Essaid} et \textbf{Imene Zaidi} qui étaient en thèse au laboratoire au moment du stage, pour leur accueil chaleureux, pour toutes les informations et les conseils qu'ils m'ont donnés ainsi que pour les discussions intéressantes que nous avons eu.
		\paragraph*{}
			Je remercie aussi \textbf{la direction de l'UHA} et \textbf{Lhassane Idoumghar}, directeur du département informatique et de l'équipe \acrshort{OMeGA} pour m'avoir permis de rejoindre leur personnel durant ces quelques mois. Enfin, je remercie \textbf{Mireille Jacquot} du Service des Stages de l'\acrshort{UTBM} et \textbf{Justine Lévesque}, agente de stage à l'\acrshort{UQAC} pour la gestion de mon dossier.
	\tableofcontents{}
	\listoffigures{}
	\chapter*{Introduction}\addcontentsline{toc}{chapter}{Introduction}
		\paragraph*{}
			Le cursus à l'\gls{UTBM} est entrecoupé de deux périodes de stage de 6 mois durant les 3 ans du cycle ingénieur, après le Tronc Commun. La première, après un an en branche, est le ST40 intitulé ``Stage Assistant Ingénieur'' et la seconde, un an plus tard, est le ST50 intitulé ``Projet de fin d'études - Ingénieur débutant''.
		\paragraph*{}
			Cependant j'ai décidé de participer à un programme de double-diplôme avec le \gls{DIM} de \gls{UQAC}, ajoutant une année de cours à l'\gls{UQAC} et la réalisation d'un stage.
			\begin{figure}[H]
				\centering%
				\resizebox{\textwidth}{!}{\import{../figures/}{cursus_utbm_uqac.tex}}%
				\caption{Cursus à l'\acrshort{UTBM} avec l'\acrshort{UQAC}}%
				\label{fig:cursus_utbm_uqac}%
			\end{figure}
		\paragraph*{}
			Comme représenté sur la figure \ref{fig:cursus_utbm_uqac}, l'année de cours a eu lieu après mon semestre INFO4, sur l'année scolaire 2018/2019. Le stage, quant à lui, sera réalisé par le cour \gls{8INF859} avec le \gls{ST50}, concerné par ce document.
		\paragraph*{}
			J'ai réalisé mon stage à l'\gls{IRIMAS}, sur le campus Illberg de l'\gls{UHA}, à Mulhouse, du 2 septembre 2019 au 7 février 2020 sous la supervision de mon maitre de stage, Laurent Moalic.
		\paragraph*{}
			L'\gls{IRIMAS} est l'unité de recherche EA 7499~\cite{RNSR_IRIMAS} de l'\gls{UHA}, c'est un institut interdisciplinaire qui rassemble tous les travaux de recherche liés aux disciplines des mathématiques, de l'informatique, de l'électronique, de l'électrotechnique, de l'automatique et du traitement du signal et de l'image à l'\gls{UHA}.
		\paragraph*{}
			Dans la première partie de ce rapport je commencerai par présenter l'\gls{UHA} et l'\gls{IRIMAS} puis dans la seconde partie, je présenterais le stage, ses objectifs et son organisation. Dans la troisième partie sera explicité le travail réalisé durant le stage et enfin dans la quatrième partie, je réaliserais une analyse du stage.
	\chapter{Présentation de l'entreprise}
		\section{L'\acrshort{UHA}}
			\subsection{Composition}
				\paragraph*{}
					L'\gls{UHA} est une université française composée de 5 campus sur 2 villes~\cite{UHA_Organisation}, les campus Illberg, Collines et Fonderie à Mulhouse et les campus Biopôle et Grillenbreit à Colmar.
				\paragraph*{}
					Elle compte 4 facultés:
					\begin{itemize}
						\item La \gls{FLSH} à Mulhouse
						\item La \gls{FSESJ} à Mulhouse
						\item La \gls{FST} à Mulhouse
						\item La \gls{FMA} à Colmar
					\end{itemize}
				\paragraph*{}
					Ainsi que 2 écoles d'ingénieurs et 2 \gls{IUT}:
					\begin{itemize}
						\item L'\gls{ENSCMu}
						\item L'\gls{ENSISA}
						\item L'\gls{IUT} de Mulhouse
						\item L'\gls{IUT} de Colmar
					\end{itemize}
				\paragraph*{}
					De plus, plusieurs services de formation et certifications lui sont ratachés (voir annexe \ref{sec:uha_formation}).
			\subsection{Formations}
				\paragraph*{}
					L'UHA compte, en 2019, 10\,366 étudiants dont 286 doctorants répartis dans plus de 170 formations dans 4 domaines~\cite{UHA_Chiffre_cles}:
					\begin{description}
						\item[10\%] Arts, Lettres, Langues
						\item[35\%] Droit, Économie, Gestion
						\item[18\%] Sciences Humaines et Sociales
						\item[37\%] Sciences, Technologie, Santé
					\end{description}
				\paragraph*{}
					Les diplômes délivrés vont du \gls{DUT} au doctorat, en passant par la licence, la licence professionnelle, le master et le diplôme d'ingénieur.
			\subsection{Recherche}
				\paragraph*{}
					L'\gls{UHA} comporte 16 laboratoires de recherche (voir annexe \ref{sec:uha_laboratories}) structurés en trois pôles:
					\begin{itemize}
						\item 6 pour le pôle Chimie, Physique, Matériaux et Environnement
						\item 2 pour le pôle Sciences pour l'Ingénieur
						\item 8 pour le pôle Sciences Humaines et Sociales
					\end{itemize}
				\paragraph*{}
					Durant le stage j'ai rejoint l'\gls{IRIMAS} qui est l'un des 2 laboratoires du pôle Sciences pour l'Ingénieur.
		\section{L'\acrshort{IRIMAS}}
			\subsection{Organisation}
				\paragraph*{}
					``L'\gls{IRIMAS} résulte de la fusion au 1er janvier 2018 du \gls{LMIA} et du laboratoire \gls{MIPS} de l'\gls{UHA}. Il regroupe l'ensemble des recherches de l'\gls{UHA} en Mathématiques, Informatique, Automatique, et Traitement du Signal et de l'Image.
				\paragraph*{}
					L’institut est organisé en trois départements:
					\begin{itemize}
						\item Mathématiques
						\item Informatique
						\item Automatique, Signal et Image
					\end{itemize}
					Il compte 75 enseignants-chercheurs permanents, une soixantaine de doctorants, une dizaine de post-docs et 5 ingénieurs/assistants-ingénieurs, avec une intense activité d’échanges académiques (plus de 80 séjours de recherche par an : stages, visiteurs scientifiques, chercheurs invités).''~\cite{UHA_IRIMAS}
				\paragraph*{}
					Les enseignants-chercheurs de l'\gls{IRIMAS} sont rattachés à 4 centres de formation de l'\gls{UHA}: l'\gls{ENSISA}, la \gls{FST}, l'\gls{IUT} de Mulhouse et l'\gls{IUT} de Colmar.
			\subsection{Équipes de recherche}
				\paragraph*{}
					Les travaux de l'\gls{IRIMAS} couvrent différentes thématiques, en recherche fondamentale aussi bien que recherche appliquée, ces travaux sont réalisés par différentes équipes, réparties dans les 3 départements~\cite{UHA_IRIMAS}:
					\begin{itemize}
						\item Département Informatique:
							\begin{itemize}
								\item Équipe \gls{OMeGA}
								\item Équipe \gls{MSD}
								\item Équipe \gls{RTe}
							\end{itemize}
						\item Département Mathématiques:
							\begin{itemize}
								\item Équipe Analyse
								\item Équipe Algèbre
							\end{itemize}
						\item Département Automatique, Signal et Image:
							\begin{itemize}
								\item Équipe \gls{MIAM}
								\item Équipe \gls{IMTI}
								\item Équipe \gls{FOTI}
							\end{itemize}
					\end{itemize}
					\begin{figure}[H]
						\centering%
						\includegraphics[width=1\textwidth]{irimas_organigramme}%
						\caption{Organigramme de l'\acrshort{IRIMAS}~\cite{IRIMAS_Organigramme}}%
						\label{fig:irimas_organigramme}%
					\end{figure}
				\paragraph*{}
					La répartition ainsi que les directeurs de ces départements sont visibles sur la figure \ref{fig:irimas_organigramme}. Durant le stage j'ai rejoint l'équipe \gls{OMeGA} du département informatique.
			\subsection{Équipe \acrshort{OMeGA}}
				\paragraph*{}
					Au sein du département informatique, l'équipe \gls{RTe} est situé à Colmar tandis que les équipe \gls{OMeGA} et \gls{MSD} sont actuellement hébergées dans l'aile sud du bâtiment Lumière de l'\gls{ENSISA} sur le campus Illberg à Mulhouse (point I sur le plan de l'annexe \ref{sec:uha_illberg_map}).
				\paragraph*{}
					L'équipe est sous la responsabilité de Lhassane Idoumghar (qui dirige aussi le département informatique) et est composée d'une douzaine de chercheurs et enseignants-chercheurs et de 5 doctorants qui travaillent essentiellement sur les axes de recherche ``Optimisation par Métaheuristiques'' et ``Algorithmique et Modélisations''.
				\paragraph*{}
					Une métaheuristique est un ensemble de concepts pouvant être utilisés pour définir des méthodes heuristiques pouvant être appliquées à un large éventail de problèmes différents, souvent des problèmes d’optimisation difficile. En d'autres termes, une métaheuristique peut être considérée comme un cadre algorithmique général qui peut être appliqué à différents problèmes d'optimisation avec relativement peu de modifications pour les adapter à un problème spécifique.
				\paragraph*{}
					``Le premier groupe de travail s’intéresse à l’algorithmique pour l’intelligence artificielle en développant de nouveaux algorithmes hybrides (basés sur des métaheuristiques à base d’agents intégrant des méthodes d’apprentissage) pour résoudre des problèmes d'optimisation mono-objectif ou multi-objectif de natures continues, discrètes et combinatoires, avec la prise en compte de l’aspect dynamique dans certains cas. L’adaptation et l’implémentation de ces futurs algorithmes hybrides sur des clusters hétérogènes de machines GPU sont également étudiées.''~\cite{IRIMAS_OMeGA}
				\paragraph*{}
					``L’activité du second groupe s’inscrit dans le thème d’analyse de données visuelles et géométriques (images, vidéo, nuages de points, courbes, etc.). Dans le cadre de cette thématique, ils développent des algorithmes et des concepts pour résoudre des problèmes de nature géométrique qui apparaissent dans de nombreuses disciplines de l’informatique [\ldots] ainsi que des techniques d’extraction d’information dans des données visuelles pour répondre à des problématiques de visualisation et d’analyses de scènes.''~\cite{IRIMAS_OMeGA}
				\paragraph*{}
					Laurent Moalic, qui a proposé le sujet de ce stage (détaillé dans la partie \ref{sec:presentation_stage}), est membre de l'équipe \gls{OMeGA}, le sujet proposé, ainsi que ces travaux en général, s’inscrivent dans l'axe Optimisation par Métaheuristiques du laboratoire.
	\printbibliography[heading=bibintoc]{}
	\printglossary[type=\acronymtype,nogroupskip=true,title=Lexique,toctitle=Lexique]{}
	\chapter*{Annexes}\addcontentsline{toc}{chapter}{Annexes}\markboth{Annexes}{}
		\setcounter{section}{0}
		\renewcommand{\thesection}{\Alph{section}}
		\renewcommand{\theHsection}{appendixsection.\Alph{section}}
		\section{Centres de formations de l'\acrshort{UHA}}\label{sec:uha_formation}
			\paragraph*{}
				%!TEX encoding = UTF-8 Unicode

\begin{tabularx}{\linewidth}{lX}
	\toprule
	\multicolumn{2}{c}{Unités de Formation et de Recherche}\\
	\midrule
	\acrshort{FLSH} & \acrlong{FLSH} de Mulhouse\\
	\acrshort{FSESJ} & \acrlong{FSESJ} de Mulhouse\\
	\acrshort{FST} & \acrlong{FST} de Mulhouse\\
	\acrshort{FMA} & \acrlong{FMA} de Colmar\\
	\bottomrule
\end{tabularx}

				\hfill\\
			\paragraph*{}
				%!TEX encoding = UTF-8 Unicode

\begin{tabularx}{\linewidth}{lX}
	\toprule
	\multicolumn{2}{c}{Instituts et Écoles d’ingénieur}\\
	\midrule
	\acrshort{ENSCMu} & \acrlong{ENSCMu} de Mulhouse\\
	\acrshort{ENSISA} & \acrlong{ENSISA} de Mulhouse\\
	\acrshort{IUT} de Mulhouse & \acrlong{IUT} de Mulhouse\\
	\acrshort{IUT} de Colmar & \acrlong{IUT} de Colmar\\
	\bottomrule
\end{tabularx}

				\hfill\\
			\paragraph*{}
				%!TEX encoding = UTF-8 Unicode

\begin{tabularx}{\linewidth}{lX}
	\toprule
	\multicolumn{2}{c}{Services de formation et certifications}\\
	\midrule
	\acrshort{CFAU} & \acrlong{CFAU}\\
	\acrshort{CUFEF} & \acrlong{CUFEF}\\
	\acrshort{SERFA} & \acrlong{SERFA}\\
	\acrshort{CLAM} & \acrlong{CLAM}\\
	\bottomrule
\end{tabularx}

				\hfill\\
		\newpage\section{Laboratoires de l'\acrshort{UHA}}\label{sec:uha_laboratories}
			\paragraph*{}
				%!TEX encoding = UTF-8 Unicode

\begin{tabularx}{\linewidth}{lX}
	\toprule
	\multicolumn{2}{c}{Pôle Chimie, Physique, Matériaux et Environnement}\\
	\midrule
	\acrshort{GRE} & \acrlong{GRE}\\
	\acrshort{IPHC} & \acrlong{IPHC}\\
	\acrshort{S2M} & \acrlong{S2M}\\
	\acrshort{LIMA} & \acrlong{LIMA}\\
	\acrshort{LPIM} & \acrlong{LPIM}\\
	\acrshort{LVBE} & \acrlong{LVBE}\\
	\bottomrule
\end{tabularx}

				\hfill\\
			\paragraph*{}
				%!TEX encoding = UTF-8 Unicode

\begin{tabularx}{\linewidth}{lX}
	\toprule
	\multicolumn{2}{c}{Pôle Sciences pour l'Ingénieur}\\
	\midrule
	\acrshort{IRIMAS} & \acrlong{IRIMAS}\\
	\acrshort{LPMT} & \acrlong{LPMT}\\
	\bottomrule
\end{tabularx}

				\hfill\\
			\paragraph*{}
				%!TEX encoding = UTF-8 Unicode

\begin{tabularx}{\linewidth}{lX}
	\toprule
	\multicolumn{2}{c}{Pôle Sciences Humaines et Sociales}\\
	\midrule
	\acrshort{ARCHIMEDE} & \acrlong{ARCHIMEDE}\\
	\acrshort{BETA} & \acrlong{BETA}\\
	\acrshort{CERDACC} & \acrlong{CERDACC}\\
	\acrshort{CREGO} & \acrlong{CREGO}\\
	\acrshort{CRESAT} & \acrlong{CRESAT}\\
	\acrshort{ILLE} & \acrlong{ILLE}\\
	\acrshort{LISEC} & \acrlong{LISEC}\\
	\acrshort{SAGE} & \acrlong{SAGE}\\
	\bottomrule
\end{tabularx}

				\hfill\\
		\newpage\section{Plan du campus Illberg de l'\acrshort{UHA}}\label{sec:uha_illberg_map}
			\begin{figure}[H]
				\centering%
				\includegraphics[width=0.95\textwidth]{UHA_plan_campus_Illberg}%
				\caption{Plan du campus Illberg de l'\acrshort{UHA}~\cite{UHA_Plan_acces}}%
				\label{fig:UHA_plan_campus_Illberg}%
			\end{figure}
	\makeutbmbackcover{}
\end{document}
