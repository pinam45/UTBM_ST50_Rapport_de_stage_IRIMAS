%!TEX encoding = UTF-8 Unicode

% Print number with at least 2 digits
\newcommand{\twodigits}[1]{\ifnum#1<10 0\fi\the#1}

% Print email link
\newcommand{\email}[1]{\href{mailto:#1}{#1}}

% instance
\newcommand{\instance}[1]{\texttt{#1}}

% C++
\newcommand{\Cpp}{\texorpdfstring{{C\nolinebreak[4]\hspace{-.05em}\raisebox{.4ex}{\tiny\bf ++}}}{C++}}

% Clear to the next left page
\newcommand*{\cleartoleftpage}{\clearpage\ifodd\value{page}\hbox{}\newpage\fi}

% Create 0-width tikz node inline
\newcommand{\tikzremember}[1]{\makebox[0pt][c]{\tikz[remember picture]\node(#1){};}}

% Custom label
\makeatletter
\newcommand{\customlabel}[2]{\def\@currentlabel{#2}\label{#1}}
\makeatother

% Table plot
\newcommand{\tableplot}[1]{%
	\raisebox{-2pt}{%
		\resizebox{25pt}{9pt}{%
			\begin{tikzpicture}%
				\begin{axis}[%
					ybar,%
					width=1000pt,%
					%height=55pt,%
			  		bar width=0.75,%
					xmin=0,%
					ymin=0,%
					enlargelimits=0,%
			  		axis lines=none,%
			  		xtick=\empty,%
			  		ytick=\empty,%
				]%
				\foreach \x [count=\xi] in {#1}
				{
					\ifthenelse{\isodd\xi}{
						\addplot+[
							black,
							fill=black,
							bar shift=-0.5
						] coordinates{(\xi,\x)};
					}{
						\addplot+[
							gray,
							fill=gray,
							bar shift=-0.5
						] coordinates{(\xi,\x)};
					}
				}
				\end{axis}
			\end{tikzpicture}%
		}%
	}%
}
